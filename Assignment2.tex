\documentclass[10pt]{article}

\usepackage[letterpaper,left=0.5in,right=0.5in,top=1in,bottom=1in]{geometry}

\usepackage[T1]{fontenc}
\usepackage[utf8]{inputenc}
\usepackage{lmodern}

\usepackage[activate={true,nocompatibility},final,tracking=true,kerning=true,spacing=true,factor=1100,stretch=10,shrink=10]{microtype}
\microtypecontext{spacing=nonfrench}
\usepackage{xspace}
\usepackage{amssymb,amsfonts,amsmath}
\usepackage{lipsum,xcolor}
\usepackage{graphicx}
\usepackage{float,caption,subcaption,wrapfig}
\usepackage{tcolorbox}
\usepackage{listings}
\usepackage{courier}
\usepackage[english]{babel}
\usepackage{textcomp}
\usepackage{csquotes}
\usepackage{siunitx}
\sisetup{mode=text,
         group-separator={,},
         detect-all,
         binary-units,
         list-units = single,
         range-units = single,
         range-phrase = --,
         per-mode = symbol-or-fraction,
         list-final-separator = {, and }
}

\lstset{basicstyle=\ttfamily,
        breaklines=true,
        numbersep=-8pt,
        numberstyle=\small,
        numbers=right,
        frame = single, 
        showstringspaces=false,    
        keywordstyle=\color{blue}\bf,
        commentstyle=\color{darkgray},
        stringstyle=\color{purple}\bf,
  }

\DeclareSIUnit\atm{atm}
\DeclareSIUnit\bar{bar}

\headheight = 13.6pt
\usepackage{fancyhdr}
\pagestyle{fancy}

\lhead{ME 568 Sp2020}
\chead{Assignment \#1}
\rhead{Due --- pm, 23 April 2020}

\rfoot{Submitted by: Paige Lorson}
\tcbset{width=(\linewidth-2mm),before=,after=\hfill,colframe=black,colback=white,}
\newcommand{\volume}{{\ooalign{\hfil$V$\hfil\cr\kern0.08em--\hfil\cr}}}
\newenvironment{Solution}
    {\textbf{Solution:}
    
    \vspace{5mm}
    \begin{tcolorbox}
    }
    {
    \end{tcolorbox}
    \vspace{5mm}
    % \newpage
    }


\begin{document}

% \noindent\textbf{Note:} 

\begin{enumerate}

%%%%%%%%%%%%%%%%%%%%%%%%%%%%%%%%%%%%%%%%%%%%%%%


\item \textbf{Tennekes and Lumley Problem 2.3} A certain amount of hot fluid is released in a turbulent flow with characteristic velocity $u$ and characteristic length $\ell$. The temperature of the patch is higher than the ambient temperature, but the density difference and the effects of buoyancy may be neglected. Estimate the rate of spreading of the patch of hot fluid and the rate at which the maximum temperature difference decreases. Assume that the size of the patch at the time of release is much smaller than $\ell$ and much larger than the Kolmogorov microscale $\eta$. The use of an eddy diffusivity is appropriate, but the choice of the velocity and length scales that are needed to form an eddy diffusivity requires careful thought, in particular as long as the size of the patch remains smaller than the length scale
C. In this context, a review of Problem 1.3 will be helpful.

\begin{Solution}

\end{Solution}
\newpage

\item \textbf{Tennekes and Lumley Problem 3.1} Estimate the characteristic velocity of eddies whose size is equal to the Taylor microscale $\lambda \text { (see Problem } 1.3) .$ Use this estimate to show that eddies of this size contribute very little to the total dissipation rate.


\begin{Solution}


\end{Solution}
% \newpage 
\item \textbf{Tennekes and Lumley Problem 3.2} Experimental evidence suggests that the dissipation rate is not evenly distributed over the volume occupied by a turbulent flow. The distribution of the dissipation rate appears to be intermittent, with large dissipation rates occupying a small volume fraction. Make a model of this phenomenon by assuming that all of the dissipation occurs in thin vortex tubes (diameter $\eta$. characteristic velocity $a=\left[\frac{1}{3} \overline{u_{i} \mu_{i}}\right]^{1 / 2}$ ). What is the volume fraction occupied by these tubes? Verify if the approximate vorticity budget $(3.3 .62)$ indeed holds for these vortex tubes.


\begin{Solution}


\end{Solution}
% \newpage 

\item Derive the TKE equation using the prescription given in class


\begin{Solution}


\end{Solution}
% \newpage 
\end{enumerate}
\end{document}
